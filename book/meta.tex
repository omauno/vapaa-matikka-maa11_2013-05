\renewcommand{\kurssinTunnus}{MAA11}
\renewcommand{\kurssinNimi}{\maaXI}
\renewcommand{\sitaatti}{''How often have I said to you that when you have eliminated the impossible, whatever remains,
\textit{however improbable}, must be the truth?'' -- Sherlock Holmes}
\renewcommand{\sitaatinLahde}{The Sign of the Four (1890), Chap. 6, p. 111}
\renewcommand{\kirjanVersio}{1.00}
\varitfalse
\mikrofalse
\versiofalse

% painettu
%\renewcommand{\kirjanPainopaikka}{\vspace*{25pt} Picaset Oy, Helsinki, 2013}
%\renewcommand{\kirjanVastaavuus}{Tätä painettua kirjaa vastaa sähköinen julkaisu.}

% web
\renewcommand{\kirjanPainopaikka}{}
\renewcommand{\kirjanVastaavuus}{Tätä sähköistä julkaisua vastaa painettu kirja.}

\renewcommand{\metasivu}{
\vspace*{\fill}
\begin{flushleft}
    \sffamily
    Jos olet kiinnostunut Vapaa matikka -sarjan kirjoittamisesta,
    voit tehdä pull requestin muutoksillesi GitHub-palvelussa tai
    osallistua yhdistyksen toimintaan, lisätietoja:
    \href{mailto:vapaamatikka@avoimetoppimateriaalit.fi}{vapaamatikka@avoimetoppimateriaalit.fi}. \\
    \vspace*{25pt}
    Erityisesti tätä kirjaa koskevaa palautetta voi lähettää Antti Rasilalle,
    \href{mailto:antti.rasila@iki.fi}{antti.rasila@iki.fi}. \\
    \vspace*{25pt}
    2., korjattu painos 2013 \\
    \vspace*{25pt}
    \kirjanVastaavuus \\
    ISBN 978-952-7010-00-6 (painettu kirja) \\
    ISBN 978-952-7010-01-3 (sähköinen julkaisu) \\
    \vspace*{25pt}
    YKL 12, 51.1 \\
    UDK 16, 511 \\
    \kirjanPainopaikka
\end{flushleft}
}
